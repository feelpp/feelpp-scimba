\documentclass[11pt]{article}
\usepackage[utf8]{inputenc}
\usepackage{amsmath,amssymb,amsthm}
\usepackage{enumitem}
\usepackage{hyperref}
\usepackage{listings}
\usepackage{xcolor}

\definecolor{mygreen}{RGB}{28,172,0} 
\definecolor{mylilas}{RGB}{170,55,241}

\lstset{
    language=C++,
    basicstyle=\ttfamily\footnotesize,
    keywordstyle=\color{blue},
    stringstyle=\color{mylilas},
    commentstyle=\color{mygreen},
    breaklines=true,
    showstringspaces=false
}

\title{Coupling Feel++ and SciMBA for Advanced PDE Solving:\\ Project Proposals}
\author{Your Name}
\date{\today}

\begin{document}

\maketitle

\section{Introduction}
This document proposes several projects aimed at coupling the finite element library \textbf{Feel++} with the scientific machine learning library \textbf{SciMBA} (which includes techniques such as Physics-Informed Neural Networks (PINNs) and Neural Operators). These projects are designed for a master’s semester project and focus on establishing a direct in-memory interface to build a \( P_c^k \) Lagrange interpolant from SciMBA data without resorting to file I/O.

\section{Direct Interface: Building the \( P_c^k \) Lagrange Interpolant}

\subsection{Motivation}
By directly coupling Feel++ with SciMBA through an in-memory interface:
\begin{itemize}
    \item \textbf{Performance \& Flexibility:} Eliminates the overhead of file I/O, enabling real-time data exchange.
    \item \textbf{Tight Coupling:} Allows Feel++ to build a Lagrange interpolant from SciMBA data on the fly, which is useful when using SciMBA predictions (e.g., as boundary data or initial conditions) in the FEM simulation.
\end{itemize}

\subsection{How It Works}
\begin{enumerate}[label=\arabic*.]
    \item \textbf{Data Exchange:} Utilize a binding library (e.g., PyBind11) to expose C++ functions to Python (or vice versa) so that data (e.g., as a \texttt{std::vector<double>}) can be passed directly.
    \item \textbf{Interpolation:} In Feel++, construct a finite element space of order \( k \) and build a \( P_c^k \) Lagrange interpolant such that
    \[
    u_h(\xi_i) = u_{\text{scimba}}(\xi_i),
    \]
    where \(\xi_i\) are the interpolation nodes.
    \item \textbf{Application:} Use the interpolant as an initial guess, source term, or to enforce boundary conditions in the FEM simulation.
\end{enumerate}


\section{Project Proposals}

\subsection{Project 1: PINNs vs FEM (Benchmark Comparison)}
\textbf{Goal:}  
Train a PINN using SciMBA to solve a simple PDE and compare its results to Feel++'s FEM solution.

\textbf{Example: Poisson Equation}
\[
-\Delta u = f \quad \text{in } \Omega, \quad u = 0 \text{ on } \partial\Omega,
\]
with a source term, for instance,
\[
f(x,y) = 2\pi^2 \sin(\pi x) \sin(\pi y).
\]

\textbf{Steps:}
\begin{enumerate}[label=\arabic*.]
    \item Solve the Poisson problem using Feel++.
    \item Generate the solution field \( u(x,y) \) on a grid.
    \item In Python, use SciMBA to define and train a PINN with the same PDE loss on collocation points.
    \item Directly pass the PINN output to Feel++ via the in-memory interface, build the \( P_c^k \) interpolant, and compare the PINN result with the FEM solution (e.g., using L2 error norm and visual plots).
\end{enumerate}

\textbf{Tools:}
\begin{itemize}
    \item Feel++ toolbox (e.g., \texttt{feelpp\_toolbox\_heat})
    \item SciMBA PINN tutorials
    \item Matplotlib for visualization
\end{itemize}

\subsection{Project 2: Neural Operator Surrogate for FEM}
\textbf{Goal:}  
Develop a surrogate model using a Neural Operator that replicates Feel++ simulations over a range of parameters.

\textbf{Example: Parametric Diffusion Equation}
\[
-\nabla \cdot (a(x,y;\mu) \nabla u) = f(x,y),
\]
where \(\mu\) controls the diffusion coefficient \(a\).

\textbf{Steps:}
\begin{enumerate}[label=\arabic*.]
    \item Generate a dataset by running Feel++ simulations while varying \(\mu\).
    \item Train a Fourier Neural Operator using SciMBA to learn the mapping \(\mu \mapsto u(x,y)\).
    \item For new values of \(\mu\), use the trained operator to predict the solution and pass the results directly to Feel++.
    \item Build the \( P_c^k \) Lagrange interpolant of the predicted solution and compare it against new Feel++ simulations.
\end{enumerate}

\textbf{Outcome:}  
A fast, neural operator-based surrogate model for expensive simulations.

\subsection{Project 3: Hybrid FEM + PINN Domain Decomposition}
\textbf{Goal:}  
Implement a hybrid approach by splitting the domain, solving one region with FEM (Feel++) and the other with a PINN (SciMBA), then coupling the solutions.

\textbf{Example: Heat Equation on \([0,1]\)}
Split the domain at \( x = 0.5 \).

\textbf{Steps:}
\begin{enumerate}[label=\arabic*.]
    \item Use Feel++ to solve the PDE on \([0, 0.5]\) with Neumann boundary conditions at the interface.
    \item Use SciMBA to solve the PDE on \([0.5,1]\) with a PINN.
    \item Directly exchange interface data between the two subdomains using the in-memory interface.
    \item Build the \( P_c^k \) Lagrange interpolant at the interface to ensure smooth continuity, iterating if needed.
\end{enumerate}

\textbf{Challenges:}
\begin{itemize}
    \item Matching boundary conditions at the interface.
    \item Efficient data exchange between Python and C++.
\end{itemize}

\subsection{Project 4: Inverse Problems with PINNs}
\textbf{Goal:}  
Utilize PINNs to estimate unknown parameters from data generated by Feel++.

\textbf{Example: Poisson Equation with Unknown Source}  
Recover the unknown source \( f(x,y) \) from noisy observations of the solution \( u(x,y) \).

\textbf{Steps:}
\begin{enumerate}[label=\arabic*.]
    \item Generate solution data \( u(x,y) \) using Feel++ for a known source \( f \).
    \item Add Gaussian noise to simulate measurement uncertainty.
    \item In SciMBA, train a PINN where \( f \) is modeled as a trainable neural network.
    \item Directly pass the recovered source term back into Feel++ via the in-memory interface and build the \( P_c^k \) interpolant.
    \item Compare the recovered source to the ground truth.
\end{enumerate}

\subsection{Additional Project Ideas}

\subsubsection*{Project 5: Adaptive Mesh Refinement Driven by SciMBA Predictions}
\textbf{Goal:}  
Use SciMBA predictions to inform adaptive mesh refinement in Feel++.

\textbf{Workflow:}
\begin{enumerate}[label=\arabic*.]
    \item Run a coarse FEM simulation using Feel++.
    \item Use SciMBA (via PINNs or Neural Operators) to predict regions with steep gradients.
    \item Directly pass these predictions to Feel++.
    \item Build the \( P_c^k \) interpolant to map predictions onto the FEM mesh, refining the mesh where necessary.
\end{enumerate}

\subsubsection*{Project 6: Multiphysics Coupling with Direct Interface}
\textbf{Goal:}  
Couple two different physical phenomena solved by different methods through direct data exchange.

\textbf{Workflow:}
\begin{enumerate}[label=\arabic*.]
    \item Solve one physical problem (e.g., fluid flow) with SciMBA’s PINN.
    \item Solve a related problem (e.g., structural deformation) with Feel++.
    \item Exchange interface data directly using the \( P_c^k \) interpolant.
    \item Iterate to ensure consistency across the coupled simulation.
\end{enumerate}

\section{Conclusion}
These projects leverage the strengths of both Feel++ and SciMBA by coupling high-fidelity FEM simulations with state-of-the-art scientific machine learning techniques. By establishing a direct in-memory interface to build the \( P_c^k \) Lagrange interpolant, file I/O overhead is eliminated, enabling rapid, iterative, and real-time hybrid simulations. This framework offers a solid foundation for a master’s project and paves the way for advanced research in numerical methods and scientific machine learning.

\end{document}