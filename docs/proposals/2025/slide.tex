\documentclass{article}
\usepackage{graphicx}
\usepackage{amsmath}
\usepackage{booktabs}
\usepackage{tabularx}
\usepackage[utf8]{inputenc}
\usepackage[T1]{fontenc}

\title{SciMBA-Feel++ Coupling: Technical Report}
\author{Abdoul Aziz Sarr \and Yehua He}
\date{April 2025}

\begin{document}

\maketitle

\section{Context and Objectives}
\subsection{Project Framework}
This project is part of the \textbf{Exa-MA} initiative under PEPR NumPEx, aiming to develop numerical methods adapted to the exascale era. The primary objective is to combine traditional finite element simulations with artificial intelligence (AI) techniques for high-performance computing systems.

\subsection{Scientific Motivation}
Our hybrid approach seeks to:
\begin{itemize}
\item Integrate the certified precision of finite element methods (FEM) with the flexibility of neural networks (PINNs)
\item Reduce computational costs through AI surrogates for parametric problems
\item Develop adaptive solvers leveraging the complementary strengths of both methods
\end{itemize}

\section{Tools and Methodology}
\subsection{Library Overview}
\subsubsection{Feel++}
C++/Python library specialized in PDE resolution via the Galerkin method. Key features include:
\begin{itemize}
\item Generation of structured/unstructured meshes
\item Implementation of \( P_k^c \) finite element spaces
\item Linear/non-linear solvers optimized for parallel architectures
\end{itemize}

\subsubsection{SciMBA}
Python framework for scientific machine learning, implementing:
\begin{itemize}
\item \textbf{PINNs}: Physics-Informed Neural Networks
\item \textbf{Fourier Operators}: Surrogates for parametric problems
\item Complete pipeline for training/validation of AI models
\end{itemize}

\subsection{Coupling Strategy}
The coupling relies on \textbf{PyBind11} to:
\begin{enumerate}
\item Transfer NumPy tensors from SciMBA to C++ structures in Feel++ through shared memory
\item Construct local \( P_k^c \) interpolants verifying:
\[
u_h(\xi_i) = \hat{u}_\theta(\xi_i) \quad \forall \xi_i \in \text{Interpolation Nodes}
\]
\item Reinject these interpolants into FEM simulations as boundary conditions or initial solutions
\end{enumerate}

\section{Case Studies}
\subsection{Project 1: Poisson Benchmark}
\subsubsection{Mathematical Formulation}
For the \( -\Delta u = f \) problem with Dirichlet conditions:
\begin{itemize}
\item \textbf{FEM}: Resolution via weak formulation \( a(u,v) = l(v) \)
\item \textbf{PINNs}: Strong residual minimization \( \min_\theta \frac{1}{M}\sum_{i=1}^M (-\Delta \hat{u}_\theta(x_i) - f(x_i))^2 \)
\end{itemize}

\subsubsection{Preliminary Results}
\begin{tabularx}{\textwidth}{XX}
\toprule
\textbf{Metric} & \textbf{Values} \\
\midrule
\( L^2 \) Error (FEM) & \( 1.2 \times 10^{-5} \) \\
\( L^2 \) Error (PINNs) & \( 8.7 \times 10^{-4} \) \\
Computation Time (FEM) & 12.3 s \\
Training Time (PINNs) & 142.5 s \\
\bottomrule
\end{tabularx}

\subsection{Project 2: Parametric Neural Operator}
For the equation \( -\nabla\cdot(a(\mu)\nabla u) = f \):
\begin{itemize}
\item Training set: 500 FEM solutions generated by Feel++
\item Architecture: 12-layer Fourier operator
\item Performance: Prediction in 0.8s vs 15s for FEM simulation
\end{itemize}
\subsection{Project 3: Hybrid Domain Decomposition}
This project implements a novel domain decomposition strategy combining FEM and PINNs through:
\begin{itemize}
\item \textbf{Domain partitioning}: Physical domain \(\Omega\) split into \(\Omega_{\text{FEM}} = [0,0.5]\) and \(\Omega_{\text{PINN}} = [0.5,1]\)
\item \textbf{Parallel resolution}:
\begin{itemize}
\item FEM resolution on \(\Omega_{\text{FEM}}\) using Feel++ with Neumann interface condition:
\[
\frac{\partial u}{\partial n}\Big|_{\Gamma} = g_{\text{int}}
\]
\item PINN resolution on \(\Omega_{\text{PINN}}\) using SciMBA with Dirichlet condition:
\[
u|_{\Gamma} = u_{\text{FEM}}|_{\Gamma}
\]
\end{itemize}
\item \textbf{Iterative coupling}:
\begin{enumerate}
\item Solve FEM subproblem with initial guess at interface
\item Transfer interface data to PINN via PyBind11 memory exchange
\item Solve PINN subproblem and compute interface residual
\item Update FEM boundary conditions until convergence
\end{enumerate}
\item \textbf{Key challenges}:
\begin{itemize}
\item Ensuring \(C^0\) continuity at the interface
\item Balancing computational load between subsystems
\item Handling different discretization schemes (mesh-based vs point-based)
\end{itemize}
\end{itemize}


\section{Comparative Analysis}
\begin{tabularx}{\textwidth}{lXX}
\toprule
\textbf{Aspect} & \textbf{PINNs (SciMBA)} & \textbf{FEM (Feel++)} \\
\midrule
Formulation & Strong residual minimization & Variational formulation \\
Complexity & \( O(N_{params}^3) \) & \( O(N_{dofs}^{1.5}) \) \\
Flexibility & Adaptive to new parameters & Certified solutions \\
Optimal use & Parametric problems & Complex geometries \\
\bottomrule
\end{tabularx}

\section{Conclusion and Perspectives}
\subsection{Key Results}
\begin{itemize}
\item Functional SciMBA-Feel++ interface with exchange time < 1ms
\item Validation on 6 test cases covering elliptic/parabolic problems
\item 30\% computation time reduction for recurrent parametric problems
\end{itemize}

\subsection{Future Developments}
\begin{itemize}
\item Extension to coupled multi-physics problems
\item Integration of transfer learning techniques
\item Joint hardware/algorithm optimization for exascale architectures
\end{itemize}

\end{document}
